%!compile = 2
\documentclass[b4j,landscape,10pt,expert,twoside,twocolumn]{tetsujsarticle}
\usepackage{鉄緑会英語科}
\pagestyle{fancy}
\usepackage{pxfonts}
\tcbuselibrary{fitting}  % 問題文の自動フィット用

%\余白設定[
%上余白=1truecm,
%下余白=上余白,
%左余白=1.414truecm,
%左右差=0truecm,
%段間隔=対称,
%段組の仕切り線の太さ=.5truept,
%ヘッダの縦幅=16truept,
%ヘッダ下端と本文上端の縦間隔=10truept,
%ヘッダと本文を仕切る線の太さ=0truept,
%本文下端とフッタ下端の縦間隔=20truept,
%本文とフッタを仕切る線の太さ=0truept,
%傍注領域の幅=0truept,
%本文端と傍注領域との間隔=0truept,
%傍注同士の縦間隔=\baselineskip,
%奇数ページの傍注位置=右,
%偶数ページの傍注位置=左,
%]
%
%\英語科基本用紙設定問題B5

% === 週ごとに変更する部分 ===
\def\学期{後期}  % 前期 or 後期
\def\週番号{13}  % 前期: 02〜22 / 後期: 01〜21(2桁で指定)
% =========================

% 問題情報を外部ファイルから読み込み
\input{../週別問題/\学期/第\週番号 週/problem.tex}

\fancyhead[R]{\sffamily 英語実戦講座 英作文B 添削用紙}
\fancyhead[L]{\sffamily \sffamily \週タイトル \テーマ}
\cfoot{}% この行が無いと勝手にフッタ中央にページ番号が出力される

\tcbset{
  mystyle/.style={
    colback=white,
    colframe=black,
    boxrule=0.5pt,
    arc=0pt,          % 角を直角に(好みで)
    valign=top,
    top=2mm, bottom=2mm, left=2mm, right=2mm
  }
}

\begin{document}

\begin{tcolorbox}[mystyle, title=【問題】, fit to height=4cm]
\問題文
\end{tcolorbox}

\begin{tcolorbox}[mystyle, title=【構成メモ】, height=5cm]

\end{tcolorbox}

\begin{tcolorbox}[mystyle, title=【解答欄】, height=12cm]
\br{1.5}
\行解答欄[行数=10,下線スタイル=\点線,行送り=10pt,末尾=\Wordscount]
\end{tcolorbox}

\newcolumn

\begin{tcolorbox}[mystyle, title=【解説・板書・メモ】, height=6cm]

\end{tcolorbox}

\begin{tcolorbox}[mystyle, title=【採点欄】, height=12cm]

\end{tcolorbox}

% スキャン基準ラベル
\begin{tikzpicture}[remember picture, overlay]
\node[anchor=north east] at ([xshift=-.4cm, yshift=-.4cm]current page.north east) {\マークシート右上ラベル};
\node[anchor=south east] at ([xshift=-.4cm, yshift=.4cm]current page.south east) {\マークシート右下ラベル};
\node[anchor=south west] at ([xshift=.4cm, yshift=.4cm]current page.south west) {\マークシート左下ラベル};
\end{tikzpicture}

\end{document}